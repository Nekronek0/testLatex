% test document to learn latex
\documentclass{report}
\usepackage[utf8]{inputenc}
\usepackage[version=4]{mhchem}
\usepackage{amsmath}
\usepackage{dcolumn}
\usepackage[table]{xcolor}
\usepackage{graphicx}

\title{Test \LaTeX{}}
\author{CY Gerritsen, MSc\\ Laboratory of Physical Organic Chemistry\\ Institute for Molecules and Materials\\ Radboud University\\ Nijmegen}
\date{31 July 2015}

\begin{document}

\maketitle
\setcounter{tocdepth}{1}
\tableofcontents{}

\chapter{Introduction}
Hello World!
\section{Cheese}
All over the world there are a lot of different cheeses, for example:
\begin{itemize}
\item Brie \item Gouda \item Langres \item \ldots

\end{itemize}
    \subsection{Brie} xxx
    \subsection{Gouda} xxxx
    \subsection{Langres} xxxxx
    
\section{Math}
\begin{align}
E &= mc^2\\
f(x) &= x^2\\
\hat{\theta} &= [\Phi^T \Phi]^{-1} \Phi \hat{y}
\end{align}

\section{Chemistry}
\paragraph{The actual chemistry}
    \ce{C12H6O12} is used to denote glucose. 
    \ce{Ca^2+} is a Calcium ion, \ce{Cl-} a chloride ion, and \ce{SO3^2-} a sulfate ion.\\
    \ce{Tg + Tr -> TgTr -> 2 Tr}\\
    \ce{\alpha I ->[Ap] I} 

\newcolumntype{a}{D{.}{.}{2.1}}
\newcolumntype{d}{D{.}{.}{1.4}}
\paragraph{The table}
In this paragraph the table is presented.\\
\begin{table}[h!]
\centering
\caption{The table with [Apep] and kflow}
\begin{tabular}{r|a|d} \rowcolor{blue!50} \multicolumn{1}{c}{} &  \multicolumn{1}{r}{$[Ap]$} & \multicolumn{1}{l}{$k_{flow}$}\\ 
 1 & 31.4 & 0.234\\ 2 & 20.9 & 0.234\\ 3 & 47.1 & 0.234\\ 4 & 7.0 & 0.234\\ 5 & 31.4 & 0.156\\ 
6 & 31.4 & 0.375\\ 7 & 31.4 & 0.0585\\ 8 & 26.2 & 0.195\\ 9 & 62.8 & 0.234\\
\end{tabular}
\end{table}
\paragraph{The Figure}
In this paragraph a figure is presented.\\
\begin{figure}[h!]
%\centering
\includegraphics[scale=0.1]{x:/Documents/MATLAB/mass_compare9_7}
\caption{Some data of fitting to dataset 9}
\label{fig:my_label}
\end{figure}
\end{document}